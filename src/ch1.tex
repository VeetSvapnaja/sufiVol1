\chapter{Sejenis Sihir yang Langka}
Uwais ditanya, ‘BAGAIMANA PERASAANMU?’

DIA MENJAWAB, ‘SEPERTI SESEORANG YANG BANGUN DI PAGI HARI DAN TIDAK TAHU APAKAH
DIA AKAN MATI DI SORE HARI.’

ORANG ITU BERKATA, ‘TAPI INI ADALAH SITUASI DARI SEMUA MANUSIA.’

UWAIS MENJAWAB, ‘YA, TAPI BERAPA BANYAK DARI MEREKA MERASAKANNYA?’


SUATU HARI seorang pelajar Muslim datang dan bertanya, “Kamu bukan Muslim, lalu mengapa engkau berbicara tentang Sufi?’ Aku katakan padanya,‘Aku bukan Muslim, itu jelas, tapi aku seorang Sufi semua sama.’

Seorang Sufi tidak perlu menjadi Muslim. Seorang Sufi bisa ada di mana saja, dalam bentuk apapun - karena sufi adalah
inti penting dari semua agama. Sufi tidak ada hubungannya dengan Islam pada khususnya. Sufisme bisa ada tanpa
Islam; Islam tidak bisa ada tanpa sufi. Tanpa sufi, Islam adalah sebuah jenazah. Hanya dengan sufi islam menjadi hidup.

Setiap kali sebuah agama hidup itu karena Sufi. Sufi berarti hubungan cinta dengan Tuhan, dengan
Yang Pokok, hubungan cinta dengan keseluruhan. Menjadi Sufi berarti bahwa Seseorang siap untuk larut ke dalam keseluruhan, 
Orang itu siap untuk mengundang Keseluran untuk datang ke dalam hatinya. Di dalam Sufi tidak ada formalitas. Sufi tidak dibatasi oleh
dogma, doktrin, keyakinan atau gereja. Kristus adalah seorang sufi, begitu juga Muhammad. Krishna adalah seorang sufi, begitu juga
Budha. Ini adalah hal pertama yang saya ingin Kamu mengerti: bahwa Sufi  adalah inti terdalam - sebagaimana
Zen ada , sebegitu pula Hadis ada. Hanya nama yang berbeda dari hubungan pokok yang sama dengan Tuhan.

Hubungan dengan Tuhan berbahaya. Hal ini berbahaya karena semakin kamu mendekat dengan Tuhan, semakin lebih
dan lebih Kamu menghilang. Dan ketika Kamu benar-benar dekat Kamu benar - benar tiada. Hal ini berbahaya
seperti bunuh diri ... tapi bunuh diri yang indah. Kematian dalam Tuhan adalah satu-satunya cara untuk benar-benar hidup.
Sebelum Kamu mati, Sebelum kamu mati secara sukarela menjadi cinta, Kamu hidup di kehidupan yang biasa-biasa saja; kamu
hambar , Kamu tidak memiliki arti apapun. Tidak ada puisi timbul di dalam hati Kamu, tiada tarian, tidak ada perayaan;
Kamu hanya meraba-raba dalam kegelapan. Kamu hidup di kehidupan minimal, Kamu tidak penuh dengan ekstase.

Kehidupan melimpah ada hanya ketika Kamu tidak ada. Kamulah halangan itu. Sufisme adalah seni menghapus
halangan antara Kamu dan Kamu, antara diri dan diri, antara bagian dan keseluruhan.

Beberapa hal tentang kata ‘Sufi’ ini. Kamus Persia kuno memiliki terjemahan untuk kata ‘Sufi’ ini... Definisinya
untuk terjemahannya ditulis dalam bentuk sajak: SUFI CHIST - SUFI, sufistik. SIAPAKAH SEORANG SUFI? SEORANG SUFI ADALAH SUFI.
Itu adalah definisi yang indah. Fenomena ini tak dapat dijelaskan. ‘Seorang Sufi adalah seorang Sufi.’  Terjemahan itu tidak berkata apa-apa
namun ia menerjemahkannya dengan sangat baik. Terjemahan itu mengatakan bahwa Sufi tidak dapat didefinisikan; tidak ada kata lain dapat mendefinisikannya, Tidak ada sinonim lain, tidak ada kemungkinan mendefinisikan kedalam bahasa, tidak ada yang lain tak dapat dijelaskan
fenomena. Kamu dapat hidup dan Kamu bisa tahu, tetapi melalui pikiran, melalui intelek, hal itu 
tidak dimungkinkan. Kamu bisa menjadi seorang sufi - yang merupakan satu-satunya cara untuk mengetahui apa itu. Kamu bisa mencicipi
realitas diri sendiri, hal itu tersedia. Kamu tidak perlu mencarinya di kamus, Kamu dapat menemuinya di Alam Semesta.

Suatu Ketika....

Seorang anak kecil sedang bermain di taman. Dia adalah seorang anak yang berbadan sangat kecil dan sangat takut akan
Anjing bulldog besar yang duduk di halaman samping rumahnya.

Suatu hari, merasa agak berani, anak kecil itu memanjat pagar, dan anjing bulldog besar itu bergegas
menghampirinya dan menjilat wajahnya. Anak itu mulai menjerit dan ibunya tiba di tempat kejadian secepat
mungkin.

‘Apakah dia menggigit Kamu, Sayang?’

‘Tidak,’ anak kecil itu merintih, ‘tapi dia mencicipiku .’

Jika engkau tidak siap untuk merasakan gigitan Sufi Kamu setidaknya bisa mencicipnya.

Dan itulah yang akan aku akan sediakan untukmu - sedikit rasa. Dan sekali engkau merasakannya bahkan hanya setetes nektar yang disebut Sufi engkau akan menjadi lebih haus lagi. Untuk pertama kalinya engkau akan mulai merasa nafsu makan yang besar akan Tuhan.

Pembicaraan ini tidak bisa menjelaskan kepadmu apakah Sufi itu - karena aku bukan filsuf. Aku juga bukan
seorang teologi.Dan aku tidak benar-benar berbicara tentang Sufisme, saya akan berbicara Sufisme. Jika engkau siap,
jika engkau siap untuk pergi ke petualangan ini, maka engkau akan mencapai rasa itu. Ini adalah sesuatu yang akan mulai terjadi di dalam hati Kamu. Ini adalah sesuatu seperti kuntum yang terbuka. Kamu akan mulai merasa sensasi khusus
di hatimu - seolah-olah sesuatu yang menjadi peringatan, terjaga di sana; seolah-olah hati telah
tertidur lama dan sekarang adalah secercah pertama dari pagi hari - dan di sana Kamu akan memiliki rasa.

Sufisme adalah jenis khusus dari sihir, sihir jenis langka. Hal ini dapat ditransfer hanya dari orang ke
orang, bukan dari buku.Sufi tidak dapat ditransfer oleh kitab suci. Hal ini sama juga seperti Zen - transmisi


diluar kata-kata. Kaum Sufi memiliki kata khusus untuk ini - mereka menyebutnya silsila. Apa yang Kaum Hindu sebut parampara mereka sebut silsila. Silsila berarti transfer dari satu hati ke hati yang lain, dari satu orang ke orang yang lain. Pengalaman itu adalah agama yang sangat, sangat  pribadi.

Kamu tidak dapat memilikinya tanpa berhubungan dengan seorang Guru yang tercerahkan - tidak ada cara lain. Kamu bisa membaca semua literatur yang ada tentang Sufisme dan Kamu akan hilang di hutan kata-kata. Kecuali jika Kamu menemukan panduan, kecuali jika Kamu jatuh cinta dengan panduan, Kamu tidak akan memiliki rasa itu.

Saya siap membawa Kamu pada perjalanan yang jauh ini, jika Kamu berani, sangat berani tuk berpetualang. Saya harap Kamu begitu - karena hanya orang - orang berani yang tertarik terhadapku. Tempat ini bukan untuk pengecut;Tempat ini bukan untuk orang-orang yang disebut orang-orang beragama; Tempat ini bukan untuk orang yang disebut orang Takut-Akan-TUhan- ini adalah tempat untuk orang-orang yang saya sebut orang yang Mencintai-Tuhan. Dan mereka memiliki kualitas yang sama sekali berbeda. Seorang yang Takut-Akan-Tuhan tidak pernah bergerak ke realitas yang lebih dalam dari agama, dia tidak bisa - karena rasa takutnya.

Kata ‘Takut-Akan-Tuhan’ sangat aneh. Jika Kamu takut akan Tuhan Kapan kamu akan menjadi penuh Cinta?
Siapa yang akan Kamu cintai? Jika Kamu bahkan tidak dapat mencintai Tuhan maka cinta tidak akan mungkin bagi Kamu juga. Jika bahkan dengan Tuhan Kamu berhubungan melalui rasa takut, maka hal ini bukan sebuah hubungan.

Tapi kita telah diajarkan untuk takut Tuhan. Bahkan, kita hanya diajarkan untuk takut akan
segala sesuatu. seluruh hidup kita adalah ketakutan, kekhawatiran, seorang pengecut - takut neraka, takut akan Tuhan, takut hukuman. Kita berlaku baik, berbudi luhur, karena kita takut. Jenis hikmat Apa yang didasarkan pada ketakutan?

Dan bagaimana Kamu bisa mencintai Tuhan jika pendekatan dasar Kamu adalah melalui rasa takut? Karena rasa takut maka cinta tidak akan pernah muncul - hal itu adalah suatu kemustahilan. Dan dari rasa cinta ketakutan tidak pernah muncul. Ketika Kamu mencintai seseorang semua ketakutan menghilang.
Dan ketika Kamu takut semua cinta menghilang. Kamu bisa membenci orang jika Kamu takut akan dia,
tetapi Kamu tidak dapat mencintainya. Berabad-abad manusia telah diajarkan untuk takut akan Tuhan dan
Hasil akhirnya adalah bahwa Nietszche harus menyatakan bahwa Tuhan sudah mati. Itu adalah hasil akhir dari
pikiran yang berorientasi rasa takut. Berapa lama Kamu dapat mentolerir Tuhan seperti ini? Berapa lama Kamu bisa tetap takut? Suatu hari nanti atau Orang lain yang akan melakukannya bahwa Kamu akan membunuhnya . Itulah yang Nietszche lakukan. Ketika ia berkata, ‘Tuhan sudah mati,’ ia juga berkata, ‘Sekarang Manusia bebas.’ ‘Tuhan telah mati dan sekarang manusia bebas.’ Kalau tidak, bagaimana Kamu bisa bebas dari Tuhan
jika Tuhan hanya sumber ketakutan? Ketakutan tidak bisa memberikan kebebasan.

Orang-orang yang datang kepadaku adalah orang-orang yang Cinta-Tuhan. Ketika saya mengatakan ‘Cinta-Tuhan’
Maksudku mereka dalam pencarian. Mereka ingin tahu. Dan mereka ingin tahu secara otentik, mereka tidak mau meminjam pengetahuan tentang hal itu. Mereka ingin memiliki rasa. Mereka ingin bertemu, mereka ingin menghadap Tuhan,mereka ingin melihat ke dalam mataNYA.

Tapi sebelum Kamu dapat mampu melihat ke dalam mata Tuhan, Kamu harus mampu melihat ke dalam mata seorang Guru. Dari sana Kamu lepas landas. Perjalanan dimulai.

Aku akan membuat diriku tersedia untuk Kamu. Sufisme adalah hanya alasan. Saya tidak akan berbicara tentang Sufi, saya akan berbicara Sufi itu sendiri. Kata ‘Sufi’ juga indah. Kata itu memiliki banyak orientasi dan semuanya indah. Dan aku tidak ingin menekankan salah satu orientasi, seperti seolah-olah ditekankan berkali - kali. Beberapa orang memilih satu orientasi, beberapa orang memilih yang lain, tetapi pemahaman saya adalah bahwa semua orientasi mereka adalah indah dan memiliki sesuatu yang istimewa. Saya menerima mereka semua

Salah satu Sufi Guru, Abul Hasam, mengatakan, ‘Sufisme pernah menjadi kenyataan tanpa nama dan sekarang Sufisme adalah nama tanpa kenyataan.’

Selama berabad-abad Sufisme ada tanpa nama. Sufisme hadir sebagai realitas. Itulah mengapa saya katan Yesus adalah seorang sufi, begitu juga Muhammad. begitu juga Mahavir dan begitu juga Krishna. Siapapun yang telah menyadari Tuhan adalah seorang Sufi. Mengapa saya katakan demikian? Cobalah untuk memahami kata ‘Sufi’ dan hal itu akan menjadi jelas untukmu.

Kata ‘Sufi’ adalah temuan baru, temuan dari Orang Jerman, dari sarjana Jerman. Tidak lebih dari
Seribu lima puluh tahun yang lalu. Dalam bahasa Arab kata itu disebut tasawwuf. Tapi keduanya berasal dari akar ‘suf’ yang berarti wol.

Mungkin terlihat aneh. Mengapa wol harus menjadi simbol dari Sufisme? Sarjana itu mengatakan bahwa itu adalah karena Kaum sufi selalu memakai jubah wol. Itu benar. Tapi kenapa? Tidak ada yang dapat menjawabnya. Kenapa mereka harus mengenakan jubah wol? Muhammad berkata dalam Alquran bahwa bahkan Musa pun mengenakan jubah wol ketika ia bertemu Tuhan. Ketika Tuhan berbicara kepadanya ia seluruh badannya ditutupi jubah wol. Tapi kenapa?

Ada simbolisme yang mendalam di dalamnya. Simbolisme berarti bahwa wol adalah pakaian hewan dan seorang sufi harus menjadi murni seperti binatang. Seorang Sufi harus mencapai kemurnian primal. Dia harus meninggalkan semua jenis peradaban, dia harus meningglkan semua jenis budaya, ia harus meninggalkan semua pengkondisian, ia harus menjadi binatang lagi. Itu mengapa simbol itu menjadi sangat signifikan.

Ketika seorang Manusia menjadi hewan ia tidak menjadi terbelakang, dia menjadi lebih tinggi. Ketika seorang Manusia menjadi hewan Manusia itu bukan hanya menjadi hewan. Hal Itu tidak mungkin. Kamu tidak dapat menjadi terbelakang. Ketika seorang Manusia menjadi hewan ia menjadi orang suci. Dia tetap sadar tapi kesadarannya tidak lebih terbebani oleh pengkondisian. Dia tidak menjadi Hindu dan tidak lagi menjadi seorang Muslim dan tidak lagi menjadi seorang Kristen. Dia selaras dengan semesta sedalam seperti hewan. Dia telah meninggalkan semua jenis filosofi, ia tidak membawa konsep apapun dalam pikirannya, pikirannya tidak dipenuhi muatan apapun. Dia tetap orang yang sama, tapi dia tidak lagi di dalam pemikiran. Hidup tanpa pemikiran - itulah arti dari jubah wol. Untuk menjadi seperti hewan yang murni, tidak untuk mengetahui apa yang baik dan apa yang buruk ... dan kemudian hikmat tertinggi muncul, ‘summum bonum’.

Ketika Kamu tahu apa yang baik dan yang buruk, dan Kamu memilih yang baik daripada yang buruk, Kamu menjadi terbelah.Bila Kamu memilih, maka ada represi. Ketika Kamu mengatakan ‘Saya akan melakukan ini. Ini harus dilakukan. Ini sebaiknya dilakukan’, ini menjadi ‘kewajiban’. Maka secara alami Kamu harus mengekang - Kamu harus mengekang yang mana Kamu telah anggap sebagai yang buruk. Dan bagian yang dikekang tetap dalam diri Kamu dan terus meracuni sistem Kamu. Dan cepat atau lambat bagian itu akan menegaskan dirinya, cepat atau lambat bagian itu akan membalas dendam. Ketika sudah tidak terbendung lagi, Kamu akan gila.

Oleh karena itu semua orang yang modern selalu di ambang kegilaan. bumi ini adalah rumah sakit jiwa besar. beberapa telah menjadi gila, beberapa telah berpotensi. Perbedaan antara Kamu dan orang gila bukanlah tentang kualitas, tapi hanya kuantitas, hanya derajat. 
Mungkin mereka telah melampaui seratus derajat dan Kamu hanya berada di sekitaranny - di sembilan puluh delapan, sembilan puluh sembilan - tetapi setiap saat situasi apapun dapat mendorong Kamu melampaui batas. Apakah Kamu tidak melihatnya? tidak bisakah Kamu amati pikiran Kamu? tidak bisakah Kamu lihat kegilaan yang berlangsung terus menerus di dalam? Hal itu terus menerus ada. Kamu menghindarinya; Kamu menyibukkan diri dalam seribu satu hal hanya untuk menghindarinya. Kamu tidak ingin melihat hal itu, Kamu ingin melupakan tentang hal itu. Hal itu terlalu menakutkan, mengerikan. Tapi itu ada - dan apakah Kamu menghindar atau tidak tapi hal itu tumbuh.
Hal itu terus mengumpulkan momentumnya. Hal itu dapat berada di puncaknya setiap saat. Setiap hal kecil bisa menjadi pemicu. Bila Kamu memilih, Kamu harus mengekang.

Hewan tidak memilih. Apa pun yang terjadi, terjadilah. hewan hanya menerimanya; penerimaannya adalah
total. Ia tahu tidak ada pilihan.

Begitu juga seorang Sufi. Sebuah Sufi tidak mengenal pilihan. Dia secara sadar tak memilih. Apapun yang terjadi ia menerima sebagai hadiah, sebagai hal yang diberikan Tuhan. Siapa dia sehingga dapat memilih Dia tidak percaya akan pikirannya, ia percaya di pikiran yang universal Itu sebabnya ketika kamu menemukan Sufi Kamu akan melihat kepolosan hewan seperti di matanya, di keberadaannya; benar - benar bebas, benar-benar sukacita, karena hanya hewan tahu - atau pohon atau batu atau bintang.

Idries Shah telah mengutuk definisi ‘sufi’ dari kata ‘suf’ - wol - persis alasan yang sama sebagaimana saya menyetujui. Dia mengatakan bahwa Sufi begitu mengerti tentang simbol bagaimana mereka dapat memilih wol
sebagai simbol? wol merupakan hewan dan Idries Shah mengatakan Sufi tidak bisa memilih hewan
sebagai simbol. Mereka adalah umat Tuhan - mengapa mereka harus memilih hewan? Dia tampaknya sangat
logis, dan alasannya mungkin memiliki daya tarik bagi banyak orang.

Tapi persis dengan alasan yang sama saya menyetujui definisi itu. Bagiku, menjadi hewan artinya menjadi
tidak bersalah, tidak tahu moralitas, tidak tahu amoralitas. Untuk menjadi hewan tidak ada penghakiman. Para Santo lebih seperti hewan daripada seperti Kamu, daripada seperti yang disebut manusia. Manusia
adalah makhluk yang tidak alami, mereka sangat tidak wajar, buatan, plastik. Seluruh hidup mereka adalah hidup dalam penipuan. Jika Kamu menyentuh wajah seseorang Kamu tidak akan pernah menyentuh wajahnya, Kamu hanya menyentuh topengnya. Dan ingatlah, tangan Kamu juga tidak nyata. tangan kamu dilapisi sarung tangan. Bahkan sepasang kekasih tidak menyentuh satu sama lain; bahkan ketika bercinta Kamu tidak murni; bahkan cinta Kamu tidak tanpa topeng. Tapi bila Kamu ingin mengasihi Tuhan Kamu harus tanpa topeng. Kamu harus meninggalkan semua tipu muslihat. Kamu harus otentik seburuk apapun Kamu
menjadi tak pemilih sebenar apapun Kamu. Dalam kemurniah yang primal Tuhan turun.

Jadi alasan Idries Shah mengutuk definisi ‘Sufi’ berasal dari kata ‘suf’ persis alasan saya menyetujuinya.

Suatu ketika....

Imam Katolik sedang berusaha untuk mengkoversi seorang Yahudi.

Dia mengatakan, ‘Yang harus Kamu lakukan adalah mengatakan tiga kali,“Saya seorang Yahudi, sekarang Saya seorang Katolik. Saya seorang Yahudi, sekarang Saya seorang Katolik. Saya seorang Yahudi, sekarang Saya seorang Katolik.”’

Orang Yahudi mengikutinya, tapi imam Katolik pikir dia sebaiknya memeriksa sendiri di hari Jumat di rumahnya.

Orang Yahudi itu sedang menggoreng ayam. ‘Sekarang, Kamu tahu Kamu tidak bisa makan ayam di hari Jumat.’

‘Oh, ya, aku bisa,’ jawabnya. ‘Aku mencelupkannya ke dalam panci tiga kali dan berkata,“Dulunya saya adalah Ayam, sekarang Saya adalah seekor ikan'.

Itulah cara kita terus hidup.

Semua agama kita adalah seperti itu - hanya verbal. tidak menyelusup ke dalam diri Kamu. Dan Kamu tahu bahwa
apapun yang Kamu katakan Kamu melakukan kebalikannya. Kamu pikir satu hal, Kamu berkata lain, dan Kamu
melakukan sesuatu yang lain. Kamu trinitas, Kamu tidak satu. Dan tiga orang itu akan di berada di tiga
arah yang berbeda. Kamu adalah kumpulan orang - maka timbulah penderitaan.
